\section{Deployment}
Since Sparkling Water is designed as a regular Spark application, its deployment cycle is strictly driven by Spark deployment strategies (refer to Spark documentation\footnote{Spark deployment guide \url{http://spark.apache.org/docs/latest/cluster-overview.html}}). Spark applications are deployed by the \texttt{spark-submit}~\footnote{Submitting Spark applications \url{http://spark.apache.org/docs/latest/submitting-applications.html}} script that handles all deployment scenarios:

\begin{lstlisting}[style=Bash]
./bin/spark-submit \
  --class <main-class> \
  --master <master-url> \
  --conf <key>=<value> \
  ... # other options \
  <application-jar> [application-arguments]
\end{lstlisting}

\begin{itemize}
	\item \texttt{--class}: Name of main class with \texttt{main} method to be executed. For example, the \texttt{water.SparklingWaterDriver} application launches H2O services.
	\item \texttt{--master}: Location of Spark cluster
	\item \texttt{--conf}: Specifies any configuration property using the format \texttt{key=value}
	\item \texttt{application-jar}: Jar file with all classes and dependencies required for application execution
	\item \texttt{application-arguments}: Arguments passed to the main method of the class via the \texttt{--class} option
\end{itemize}

\subsection{Referencing Sparkling Water}

\subsubsection{Using Fatjar}
The Sparkling Water archive provided at \url{http://h2o.ai/download} contains a Fatjar with all classes required for Sparkling Water run.

An application submission with Sparkling Water Fatjar is using the \texttt{--jars} option which references included fatjar.

\pagebreak
\begin{lstlisting}[style=Bash]
$SPARK_HOME/bin/spark-submit \
  --jars assembly/build/libs/sparkling-water-assembly-2.1.8-all.jar \
  --class org.apache.spark.examples.h2o.CraigslistJobTitlesStreamingApp \
  /dev/null
\end{lstlisting}

\subsubsection{Using Spark Package}

Sparkling Water is also published as a Spark package. The benefit of using the package is that you can use it directly from your Spark distribution without need to download Sparkling Water.

For example, if you have Spark version 2.1 and would like to use Sparkling Water version 2.1.8 and launch example \texttt{CraigslistJobTitlesStreamingApp}, then you can use the following command:

\begin{lstlisting}[style=Bash]
$SPARK_HOME/bin/spark-submit \
  --packages ai.h2o:sparkling-water-core_2.11:2.1.8,ai.h2o:sparkling-water-examples_2.11:2.1.8 \
  --class org.apache.spark.examples.h2o.CraigslistJobTitlesStreamingApp \
  /dev/null
\end{lstlisting}

The Spark option \texttt{--packages} points to coordinate of published Sparkling Water package in Maven repository.

The similar command works for spark-shell:

\begin{lstlisting}[style=Bash]
$SPARK_HOME/bin/spark-shell \
 --packages ai.h2o:sparkling-water-core_2.11:2.1.8,ai.h2o:sparkling-water-examples_2.11:2.1.8 
\end{lstlisting}

Note: When you are using Spark packages, you do not need to download Sparkling Water distribution. Spark installation is sufficient.

\newpage
\subsection{Target Deployment Environments}
Sparkling Water supports deployments to the following Spark cluster types:
\begin{itemize}
	\item{Local cluster}
	\item{Standalone cluster} 
	\item{YARN cluster}
\end{itemize}

\subsubsection{Local cluster}
The local cluster is identified by the following master URLs - \texttt{local}, \texttt{local[K]}, or \texttt{local[*]}. In this case, the cluster is composed of a single JVM and is created during application submission.

For example, the following command will run the ChicagoCrimeApp application inside a single JVM with a heap size of 5g:
\begin{lstlisting}[style=Bash]
$SPARK_HOME/bin/spark-submit \ 
  --conf spark.executor.memory=5g \
  --conf spark.driver.memory=5g \
  --master local[*] \
  --packages ai.h2o:sparkling-water-examples_2.11:2.1.8 \
  --class org.apache.spark.examples.h2o.ChicagoCrimeApp \
  /dev/null
\end{lstlisting}

\subsubsection{On Standalone Cluster}
For AWS deployments or local private clusters, the standalone cluster deployment\footnote{Refer to Spark documentation~\url{http://spark.apache.org/docs/latest/spark-standalone.html}} is typical. Additionally, a Spark standalone cluster is also provided by Hadoop distributions like CDH or HDP. The cluster is identified by the URL \texttt{spark://IP:PORT}.

The following command deploys the \texttt{ChicagoCrimeApp} on a standalone cluster where the master node is exposed on IP \texttt{machine-foo.bar.com} and port \texttt{7077}:

\begin{lstlisting}[style=Bash]
$SPARK_HOME/bin/spark-submit \ 
  --conf spark.executor.memory=5g \
  --conf spark.driver.memory=5g \
  --master spark://machine-foo.bar.com:7077 \
  --packages ai.h2o:sparkling-water-examples_2.11:2.1.8 \
  --class org.apache.spark.examples.h2o.ChicagoCrimeApp \
  /dev/null
\end{lstlisting}

In this case, the standalone Spark cluster must be configured to provide the requested 5g of memory per executor node. 

\subsubsection{On YARN Cluster}
Because it provides effective resource management and control, most production environments use YARN for cluster deployment.\footnote{See Spark documentation~\url{http://spark.apache.org/docs/latest/running-on-yarn.html}} 
In this case, the environment must contain the shell variable~\texttt{HADOOP\_CONF\_DIR} or \texttt{YARN\_CONF\_DIR} which point to Hadoop configuration directory (e.g., \texttt{/etc/hadoop/conf}).

\begin{lstlisting}[style=Bash]
$SPARK_HOME/bin/spark-submit \ 
  --conf spark.executor.memory=5g \
  --conf spark.driver.memory=5g \
  --num-executors 5 \
  --master yarn-client \
  --packages ai.h2o:sparkling-water-examples_2.11:2.1.8 \
  --class org.apache.spark.examples.h2o.ChicagoCrimeApp \
  /dev/null
\end{lstlisting}

The command in the example above creates a YARN job and requests for 5 nodes, each with 5G of memory. The \texttt{yarn-client} option forces driver to run in the client process.

\subsection{Sparkling Water Configuration Properties}

The following configuration properties can be passed to Spark to configure Sparking Water:

\textbf{Generic parameters}
\begin{table}[!ht]
\centering
{\small
\begin{tabular}{l l p{4.0cm}}
\toprule
Property name & Default value  & Description \\
\midrule
		
spark.ext.h2o.flatfile 	& true & Use flatfile (instead of multicast) for creating H2O cloud. \\  \addlinespace

spark.ext.h2o.cluster.size 	& -1 & Expected number of workers of H2O cloud. -1 automatically
detects the cluster size. This number must be equal to number of Spark workers.\\  \addlinespace

spark.ext.h2o.port.base  & 54321 & Base port used for individual H2O node configuration.\\  \addlinespace

spark.ext.h2o.port.incr  & 2 & Increment added to base port to find the next available port.\\  \addlinespace

spark.ext.h2o.cloud.timeout  & 60*1000 & Timeout (in msec) for cloud.\\  \addlinespace

spark.ext.h2o.spreadrdd.retries & 10 & Number of retries for creation of an RDD covering all existing Spark executors.\\  \addlinespace

spark.ext.h2o.cloud.name & sparkling-water- & Name of H2O cloud.\\  \addlinespace

spark.ext.h2o.network.mask & -- & Subnet selector for H2O if IP detection fails. Useful for detecting correct IP if 'spark.ext.h2o.flatfile' is false.*\\  \addlinespace

spark.ext.h2o.nthreads  & -1 & Limit for number of threads used by H2O. -1 means unlimited.\\  \addlinespace

spark.ext.h2o.disable.ga 	& false &Disable Google Analytics tracking for embedded H2O.\\

\bottomrule
\end{tabular} 
} % end of \small
\end{table}

\pagebreak
\textbf{H2O server node parameters}
\begin{table}[!ht]
\centering
{\small
\begin{tabular}{l p{3.0cm} p{3.0cm}}
\toprule
Property name & Default value  & Description \\
\midrule
		
spark.ext.h2o.node.log.level & INFO & Set H2O node internal logging level.\\  \addlinespace

spark.ext.h2o.node.log.dir  & System.getProperty ("user.dir") + File.separator + "h2ologs" or YARN container dir &	Location of h2o logs on executor machine.\\

\bottomrule
\end{tabular} 
} % end of \small
\end{table}

\textbf{H2O client parameters}
\begin{table}[!ht]
\centering
{\small
\begin{tabular}{l p{3.0cm} p{3.0cm}}
\toprule
Property name & Default value  & Description \\
\midrule

spark.ext.h2o.client.log.level & INFO & Set H2O client internal logging level (running inside Spark driver).\\  \addlinespace

spark.ext.h2o.client.log.dir & System.getProperty ("user.dir") + File.separator + "h2ologs" & Location of h2o logs on driver machine.\\  \addlinespace

spark.ext.h2o.client.web.port & -1 & Exact client port to access web UI. -1 triggers automatic search for free port starting at spark.ext.h2o.port.base.\\  
\bottomrule
\end{tabular} 
} % end of \small
\end{table}
